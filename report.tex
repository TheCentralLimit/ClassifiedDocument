\documentclass[12pt]{article}

\usepackage[style=phys,citestyle=authoryear,maxcitenames=2]{biblatex}
\addbibresource{stats-report.bib}

\usepackage[page]{appendix}

\usepackage{amsmath,amssymb,commath,mathabx,mathtools,physics,siunitx}
\usepackage{aas_macros}
\usepackage{float,subcaption}

\usepackage[margin=1in]{geometry}
\usepackage{setspace}
\doublespacing

\title{
  Working Title
}

\author{
  Jacob Lange, Chi Nguyen, Daniel Wysocki
}

\date{
  Statistical Methods for Astrophysical Sciences (ASTP-611)
  \\
  Spring 2016
}



\begin{document}

\maketitle


\begin{abstract}


\end{abstract}


\section{Introduction}
\label{sec:intro}
The detection of gravitational waves (GW) in Feburary(ref) opened a new era of astronomy; however, it is only in sync with electromagnetic astronomy that the most physics can be discovered. Electromagnetic counterparts are expected from binary sources involving matter i.e. neutron star-neutron star and neutron star-black hole. Because of this, GW detectors will work in conjunction with electromagnetic telescopes to observe a GW source. Some of these will yield weak, nearly isotropic electromagnetic counterparts and others will not. GW detectors will identify sources characterized by its chirp mass:

\begin{equation}
M_{c}=\frac{(m_{1}m_{2})^{3/5}}{(m_{1}+m_{2})}
\end{equation}

This report is organized as follows. Section 2 describes the developement of the chirp mass distribution and a electromagnetic followup classifier based on the data. Section 3 will discuss the results, and Section 4 will state our conclusions.
See Section \ref{sec:discussion} and Appendix \ref{app:example}. Example text citation is \textcite{2012ApJ...759...52D}, or in parenthesis with a page number \parencite[pg 2]{2012ApJ...759...52D}.


\section{\ldots}
\label{sec:more}





\section{Discussion}
\label{sec:discussion}




\section{Conclusions}
\label{sec:conclusions}




\printbibliography[heading=subbibliography]

\begin{appendices}

\section{Example}
\label{app:example}


\end{appendices}






\end{document}