By observing compact binary mergers with gravitational wave detectors, and following up with EM counterparts, we can begin to understand the true mass distribution of neutron stars and black holes in the Universe. In this paper, we have outlined some rudimentary methods that can be used, once a large number of detections have been made. The mass distribution estimation described in \S\ref{sec:dist} can be used to understand the true distribution, and the classifier described in \S\ref{sec:classifier} can be used to reduce resource usage, by restricting EM followup to events known to have them.

By including uncertainties in our rate estimates, these methods can be applied to a much smaller number of detections, such as the two currently reported. Of course, to have more confidence in the results, we would need a more sophisticated approach, with a more cautious treatment of uncertainty.