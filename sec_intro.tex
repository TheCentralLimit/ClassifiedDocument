The detection of gravitational waves (GW) in September(ref) opened a new era of astronomy; however, it is only in sync with electromagnetic astronomy that the most physics can be discovered. Electromagnetic counterparts are expected from binary sources involving matter i.e. neutron star-neutron star and neutron star-black hole. Because of this, GW detectors will work in conjunction with electromagnetic telescopes to observe a GW source. Some of these will yield weak, nearly isotropic electromagnetic counterparts and others will not. GW detectors will identify sources characterized by its chirp mass:

\begin{equation}
\label{chirp_mass}
\mathcal{M}_c=\frac{(m_{1}m_{2})^{3/5}}{(m_{1}+m_{2})^{1/5}}
\end{equation}

This report is organized as follows. Section \ref{sec:dist} describes the developement of the chirp mass distribution, Section \ref{sec:classifier} descibes a electromagnetic followup classifier based on the data, and Section 4 will state our conclusions.

See Section \ref{sec:discussion} and Appendix \ref{app:example}. Example text citation is \textcite{2012ApJ...759...52D}, or in parenthesis with a page number \parencite[pg 2]{2012ApJ...759...52D}.

