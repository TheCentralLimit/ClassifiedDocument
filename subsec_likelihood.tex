\subsection{The Likelihood of Fitting Parameters}
\label{subsec:likelihood}

The likelihood of each fitting parameter is:

\begin{equation}
\label{likelihood}
P(\{d\}|\lambda) = \frac{1}{n!}\prod_{k} E(x)r(x_k) p_{smooth}\exp{-\int E(x)r(x_k)dx}
\end{equation}

where $p_{smooth}$ is the smoothing prior, defined to be:

\begin{equation}
\label{smoothing_prior}
p_{smooth} = \exp{-\int \bigg[\dv[n]{(r)}{(x)}\bigg]^2 \hspace{5pt} dx}
\end{equation}

In theory, $p_{smooth}$ can be any $n^{th}$ derivative. To make our code robust, we define a function that takes n as an argument. The function then calls numpy.polynomial.polyder() to find the $n^{th}$ derivative. Next, we square the $n^{th}$ derivative and integrate it between the minimum and maximum of $x$. Here we choose $n = 3$.

The likelihood estimator is:

\begin{equation}
\label{likelihood_estimator}
\ln{P(\{d\}|\lambda)}
\end{equation}

