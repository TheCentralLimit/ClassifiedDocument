\subsection{The Likelihood of Fitting Parameters}
\label{subsec:likelihood}

For our fit, we assume a $9^{\rm th}$ degree polynomial, and use a smoothing prior to prevent over-fitting. We fit this model to the points from the histogram rate, $r(x)$, obtained in \S\ref{sec:density}. The data likelihood is
%
\begin{align}
  \ln p( &\qty{r(x_k)}_k | \lambda ) = \notag
  \\ &
 -\frac{1}{2}
  \sum_k
  \qty{\frac{[r(x_k) - F_\alpha(x_k) \lambda_\alpha]^2}{\sigma_{r(x_k)}^2}},
%
  \label{likelihood}
\end{align}
%
where $F_\alpha(x) = x^\alpha$.

$\ln p_{\rm smooth}$ is the smoothing prior, defined to be:

\begin{equation}
  \ln p_{\rm smooth}(\lambda) =
 -\gamma \int_{x_{\min}}^{x_{\max}}
  \qty[ \dv[n]{F_\alpha(x)}{x} \lambda_\alpha ]^2 \dd{x}
%
  \label{smoothing_prior}
\end{equation}

In theory, $p_{\rm smooth}$ can be any $n^{\rm th}$ derivative. To make our code robust, we define a function that takes n as an argument. The function then calls \texttt{numpy.polynomial.polynomial.polyder()} to find the $n^{\rm th}$ derivative. Next, we square the $n^{\rm th}$ derivative and integrate it between the minimum and maximum of $x$. Here we choose $n = 3$, and let $\gamma = 0.5$.